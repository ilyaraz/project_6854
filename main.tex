\documentclass[12pt]{article}

\usepackage{amsmath}
\usepackage{amssymb}
\usepackage{amsthm}
\usepackage{fullpage}
\usepackage[utf8]{inputenc}
\usepackage{hyperref}
\usepackage{ccfonts}
\usepackage[T1]{fontenc}

\renewcommand{\bfdefault}{sbc}

\newcommand{\Rbb}{\mathbb{R}}
\newcommand{\eps}{\varepsilon}
\newcommand{\set}[1]{\left\{#1\right\}}

\newtheorem{theorem}{Theorem}
\newtheorem{lemma}{Lemma}
\newtheorem{definition}{Definition}

\begin{document}
    \title{Edge and Vertex Sparsification of Undirected Graphs}
    \date{}
    \maketitle

    \section{Introduction}
    We consider two sparsification procedures for undirected graphs that respect (to some extent) values of
    flows and cuts. The first is spectral sparsification, which was initially introduced by Spielman and Teng~\cite{} and was
    inspired by a combinatorial notion of cut sparsification by Benczur and Karger~\cite{}.
    The second is vertex sparsification discovered by Leighton and Moitra~\cite{}~\cite{}.
    \subsection{Spectral Sparsification}
    Let us first define cut sparsifiers.
    \begin{definition}
        Let $G = (V, E, w)$ be an undirected weighted graph.
        We say that $G' = (V, E', w')$ is an $\eps$-cut sparsifier for $G$ if all cut values of $G'$ are within
        $1 \pm \eps$ the corresponding cut values of $G$.
    \end{definition}
    The goal is to construct a sparsifier for a fixed $\eps$ and with as few edges as possible.
    The following result is due to Benczur and Karger~\cite{}.
    \begin{theorem}
        Every graph has $\eps$-cut sparsifier with $O(n \log n / \eps^2)$ edges.
        Moreover, such sparsifier can be found in time $O(m \ldots)$\footnote{What is the exact running time?}.
    \end{theorem}
    This powerful theorem has several immediate applications:
    \begin{itemize}
        \item Using an algorithm of Orlin for finding maximum $(s,t)$-flow with running time $O(nm)$ one can find a
        $(1 + \eps)$-approximation for the maximum $(s, t)$-flow in time $\tilde{O}(m + n^2 / \eps^2)$.
        \item\footnote{What else?}
    \end{itemize}
    The notion of cut sparsification can be reformulated (and then significantly extended) using linear-algebraic tools.
    \subsection{Graph Laplacian}
    \subsection{Vertex Sparsification}
\end{document}
