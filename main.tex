\documentclass[12pt]{article}

\usepackage{amsmath}
\usepackage{amssymb}
\usepackage{amsthm}
\usepackage{fullpage}
\usepackage[utf8]{inputenc}
\usepackage{hyperref}
\usepackage{ccfonts}
\usepackage[T1]{fontenc}

\renewcommand{\bfdefault}{sbc}

\newcommand{\Rbb}{\mathbb{R}}
\newcommand{\eps}{\varepsilon}
\newcommand{\set}[1]{\left\{#1\right\}}

\newtheorem{theorem}{Theorem}
\newtheorem{lemma}{Lemma}
\newtheorem{definition}{Definition}

\begin{document}
    \title{Edge and Vertex Sparsification of Undirected Graphs}
    \date{}
    \maketitle

    \section{Introduction}
    We survey two sparsification procedures for undirected graphs that respect
    (to some extent) values of flows and cuts.
    The first is spectral sparsification,
    which was initially introduced by Spielman and Teng~\cite{ST11} and was
    inspired by a combinatorial notion of cut sparsification by Bencz\'{u}r and
    Karger~\cite{BK96}.
    The second is vertex sparsification discovered by Leighton and
    Moitra~\cite{M09}~\cite{LM10}.
    \subsection{Spectral Sparsification}
    \subsubsection{Cut Sparsifiers}
    Let us first define cut sparsifiers.
    \begin{definition}
        Let $G = (V, E, w)$ be an undirected weighted graph.
        We say that $G' = (V, E', w')$ is an $\eps$-cut sparsifier
        for $G$ if all cut values of $G'$ are within
        $1 \pm \eps$ the corresponding cut values of $G$.
    \end{definition}
    The goal is to construct a sparsifier for a fixed $\eps$ with as few edges as possible.
    The following result is due to Bencz\'{u}r and Karger~\cite{BK96}.
    \begin{theorem}
        Every graph has $\eps$-cut sparsifier with $O(n \log n / \eps^2)$ edges.
        Moreover, such sparsifier can be found in time
        $O(m \log^3 n)$ with high probability.
    \end{theorem}
    One of the immediate applications of this powerful theorem is the following.
    Using the algorithm of Goldberg and Rao~\cite{GR98} we can find a maximum $(s, t)$-flow
    (and a minimum $(s, t)$-cut)
    in time $\tilde{O}(m^{3/2})$, if all capacities are polynomially bounded.
    If we are fine with $(1 \pm \eps)$-approximation, then we can first sparsify the network,
    and then run the algorithm. The overall running time is
    $\tilde{O}(m + n^{3/2} / \eps^3)$.

    The notion of cut sparsification can be reformulated (and then significantly extended)
    using linear-algebraic tools.
    \subsubsection{Graph Laplacian}
    \subsection{Vertex Sparsification}
    \section{Spectral Sparsification}
    \section{Vertex Sparsification}
    \bibliographystyle{alpha}
    \bibliography{ir}
\end{document}
